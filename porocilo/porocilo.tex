\documentclass[11pt,a4paper]{article}

\usepackage[slovene]{babel}
\usepackage[utf8x]{inputenc}
\usepackage{graphicx}

\pagestyle{plain}

\begin{document}
\title{Poročilo pri predmetu \\
Analiza podatkov s programom R}
\author{Špela Lačen}
\maketitle

\section{Izbira teme}

Tema mojega projekta je spreminjanje deleža uporabnikov interneta skozi leta po posameznih državah, v svetovnem okvirju ter kakšno je bilo stanje leta 2014. Iz izbranih podatkov bom tudi poskušala najt povezavo med  GDPpc v posamezni državi ter deležem uporabnikov interneta in tudi če višji delež  vpliva na višjo pričakovano življenjsko dobo.
\\ Cilj tega projekta je spoznavanje programa R skozi analizo podatkov.

\\ (Povezava do raziskave na kaj vse vpliva dostop do interneta: \item \url{http://www.mckinsey.com/insights/high_tech_telecoms_internet/internet_matters}

\newpage

\section{Obdelava, uvoz in čiščenje podatkov}
V svoj program sem uvozila 3 tabele  v obliki .csv ter 3 iz spletnih strani. Več o uvozu posamezne tabele je v datoteki "uvoz.r".
Nekaj tabel sem v R-ju sestavila sama s pomočjo uvoženih datotek.

\\ Podatke sem pridobila iz naslednjih spletnih strani:
\begin{itemize} 
\item \url{http://data.worldbank.org/indicator/IT.NET.USER.P2}
\item \url{http://www.internetlivestats.com/internet-users-by-country/}
\item \url{http://www.internetlivestats.com/internet-users/#trend}
\item \url{http://www.internetworldstats.com/stats.htm}
\item \url{http://databank.worldbank.org/data/views/reports/tableview.aspx?isshared=true
\item \url{http://www.nationmaster.com/country-info/stats/Health/Life-expectancy-at-birth%2C-total/Years#map}





\\ Iz tabel sem narisala grafe, ki prikazujejo...

Na koncu sem grafe izvozila v pdf datoteki:


\section{Analiza in vizualizacija podatkov}
V tretji fazi sem podatke uvožene v drugi fazi prikazala na zemljevidu, zato sem v R uvozila zemljevid sveta v obliki shp.
 \\
\includegraphics{../slike/povprecna_druzina.pdf}

\section{Napredna analiza podatkov}

\includegraphics{../slike/naselja.pdf}

\end{document}
