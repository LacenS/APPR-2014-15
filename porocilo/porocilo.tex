\documentclass[11pt,a4paper]{article}

\usepackage[slovene]{babel}
\usepackage[utf8x]{inputenc}
\usepackage{graphicx}
\usepackage{hyperref}
\usepackage{rotating}
\usepackage{amsthm}
\usepackage{pdfpages}
\usepackage{url}
\usepackage{hyperref}
\usepackage{breakurl}
\usepackage{animate}



\pagestyle{plain}
\begin{document}


\author{Špela Lačen}
\title{Poročilo pri predmetu \\
Analiza podatkov s programom R}
\maketitle

\section{Izbira teme}

Tema mojega projekta je spreminjanje deleža uporabnikov interneta skozi leta po posameznih državah in v svetovnem okvirju. Iz izbranih podatkov bom poskušala poiskat povezavo med  GDPpc, pričakovano življenjsko dobo ter deležem uporabnikov interneta. 

\bigskip 
Cilj projekta je spoznavanje programa R skozi analizo podatkov.



\section{Obdelava, uvoz in čiščenje podatkov}

V svoj program sem uvozila 3 tabele  v obliki \verb|.csv| ter 4 iz spletnih strani. Tu mi je največ težav povzročalo to, da R striktno ločuje tipe podatkov\-(\verb|matrika, data.frame...|) in da sem ugotovila s kakimi tabelami delam. Veliko časa sem tudi porabila, da sem preimenovala stolpce in vrstice. To se je v kasnejših fazah, sploh pri številskih označbah(na primer za posamezna leta) izkazalo za slabo potezo, saj je pri uporabi podatkov iz tabel pomembno, da stolpci in vrstice nimajo številskih imen. 

Več o uvozu posamezne tabele je še v datoteki \verb|uvoz/uvoz.r| v komentarjih. 

\smallskip
Nekaj tabel sem v R-ju zaradi kasnejše analize sestavila sama s pomočjo uvoženih datotek.

\bigskip
Podatke sem pridobila iz naslednjih spletnih strani:
\begin{itemize} 
\item \url{http://data.worldbank.org/indicator/IT.NET.USER.P2}
\item \url{http://www.internetlivestats.com/internet-users-by-country/}
\item \url{http://www.internetlivestats.com/internet-users/#trend}
\item \url{http://www.internetworldstats.com/stats.htm}
\item \url{http://databank.worldbank.org/data/views/reports/tableview.aspx?isshared=true}
\item \burl{http://www.nationmaster.com/country-info/stats/Health/Life-expectancy-at-birth%2C-total/Years#map}

\end {itemize}

\bigskip
Iz tabel sem nato narisala grafe. Ta del se mi zdi najbolj zahteven saj obstaja ogromno funkcij za risanje tabel in vsaka ima svoje specifike v kakšni obliki moraš podat podatke.

\bigskip
Prvi graf je stolpični in prikazuje prvih 10 držav, ki predstavljajo največji delež uporabnikov interneta po svetu. Prvi stolpec prikazuje kolikšen delež prebivalcev ima dostop do interneta, drugi pa kolikšen delež uporabnikov interneta predstavljajo.

Tu se mi je zdelo najbolj zanimivo, da recimo Kitajci predstavljajo več kot 20 \% vseh uporabnikov interneta, med tem, ko jih ima manj kot 50 \% dostop do interneta.

\includepdf[pages={1}, scale=.7]{../slike/graf1.pdf}

\newpage
Drugi graf sem naredila iz svoje tabele. Prikazuje pa število držav razvrščenih v skupine po deležu uporabnikov interneta med leti 2000-2014. Iz te tabele se lepo vidi, kako število držav, v katerih ima manjši delež prebivalcev dostop do interneta pada, število držav z višjim deležem uporabnikov interneta pa narašča.
S to tabelo sem imela kar veliko težav saj nisem vedela, da moram funkciji \verb|barplot|, tabelo podat kot \verb|as.matrix|.

\includegraphics[width=\textwidth]{../slike/graf2.pdf}

\newpage
Tretji graf prikazuje naraščanje števila uporabnikov interneta med leti 2000-2014. Narejen je s funkcijo \verb|qplot|, ki ima res ogromno možnosti prikaza podatkov, vendar še nisem osvojila v kakšni obliki moram podat podatke. 
\-
\includegraphics[width=\textwidth]{../slike/graf3.pdf}

\newpage
Četrti graf sem naredila iz četrte tabele. Narisala sem ga s pomočjo funkcije \verb|3Dpie|, prikazuje pa delež uporabnikov interneta v letu 2014 po posameznih kontinentih.

\includegraphics[width=\textwidth]{../slike/graf4.pdf}

\newpage


\section{Analiza in vizualizacija podatkov}
V tretji fazi sem uvozila zemljevid sveta v obliki shp iz spletne strani \href{http://www.naturalearthdata.com/http//www.naturalearthdata.com/download/50m/cultural/ne_50m_admin_0_countries.zip}{Natural Earth Data}. Nato sem morala preuredit tabele, da sem jih lahko uporabila za risanje zemljevida. Funkcija \verb|preuredi| zaradi prevelikih razlik ni delovala, tako da sem morala imena tabel spremeniti in nato uporabiti funkcijo \verb|match| (ob pomoči asistenta). Pri risanju zemljevida mi je funkcija \verb|spplot| ves čas povzročala težave, tako da so(zaenkrat) vsi zemljevidi narisani s funkcijo \verb|plot|. 

Prvi zemljevid je sestavljen iz 3 zemljevidov, ki prikazujejo delež uporabnikov interneta v letih 2000, 2007 in 2014. Iz njih se lepo vidi naraščanje števila uporabnikov.

\newpage

\begin{figure}[htp] \centering{
\includepdf[pages={-}, nup=1x3]{../slike/zemljevid1.pdf}}

\end{figure}  


\newpage

Na drugem zemljevidu sem uporabila podatke o deležu uporabnikov interneta v letu 2014, na prvem sem z rumenimi pikami označila države, ki spadajo v " High income group ", z roza pa države, ki spadajo v " Low income group ". Iz zemljevida je razvidno, da so rumene pike na območjih z višjim deležem uporabnikov interneta, med tem, ko so roza pike večinoma v državah z nižjim procentom uporabnikov interneta.


\begin{figure}[h!] 
\begin{center}
 \includepdf[]{../slike/zemljevid2.pdf}
 \end{center}
\end{figure}

\newpage

Na tretjem zemljevidu, ki je prav tako sestavljen iz dveh delov, sem  želela primerjat če je kakšna povezava med višjo pričakovano življenjsko dobo ter višjim deležem uporabnikov interneta. Podatki so iz leta 2011. \\

\begin{figure}[h!]
\begin{center}
  \includepdf[pages={1-2},nup=1x2, scale=.6]{../slike/zemljevid3.pdf}
  \end{center}
\end{figure}

\newpage
Na četrtem zemljevidu je pa vidna primerjava med višino GDP pc ter deležem uporabnikov interneta.

\begin{figure}[h!]
\begin{center}
  \includepdf[pages={1},nup=1x2, scale=.65]{../slike/zemljevid4.pdf}
  \end{center}
\end{figure}

\newpage
\section{Napredna analiza podatkov}
\begin{center}
\begin{figure}
  \animategraphics[controls, loop, width=1.2\linewidth]{1}{../slike/animacija}{}{}
\end{figure}

\end{center}


\end{document}









